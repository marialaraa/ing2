\documentclass[a4paper, 11pt]{article}
\usepackage{amsmath}
\usepackage{amsfonts}
\usepackage{amssymb}
\usepackage{caratula}
\usepackage[spanish, activeacute]{babel}
\usepackage[usenames,dvipsnames]{color}
\usepackage[width=15.5cm, left=3cm, top=2.5cm, height= 24.5cm]{geometry}
\usepackage{graphicx}
\usepackage[utf8]{inputenc}
\usepackage{listings}
\usepackage[all]{xy}
\usepackage{multicol}
\usepackage{subfig}

\usepackage{cancel}
\usepackage{float}
\usepackage{xcolor}
\usepackage{color,hyperref}

\usepackage{multirow} % para las tablas


%%%%%%%%%%%%%% ALGUNAS MACROS %%%%%%%%%%%%%%
% For \url{SOME_URL}, links SOME_URL to the url SOME_URL
\providecommand*\url[1]{\href{#1}{#1}}

% Same as above, but pretty-prints SOME_URL in teletype fixed-width font
\renewcommand*\url[1]{\href{#1}{\texttt{#1}}}

% Comando para poner el simbolo de Reales
\newcommand{\real}{\hbox{\bf R}}

\providecommand*\code[1]{\texttt{#1}}

%uso: \ponerGrafico{file}{caption}{scale}{label}
\newcommand{\ponerGrafico}[4]
{\begin{figure}[H]
	\centering
	\subfloat{\includegraphics[scale=#3]{#1}}
	\caption{#2} \label{fig:#4}
\end{figure}
}

%\renewcommand{\algorithmiccomment}[1]{\hfill #1}

%%%%%%%%%%%%%%%%%%%%%%%%%%%%%%%%%%%%%%%%%%%%

\materia{Ingeniería de Software II}

\titulo{Big Tiza}
%\fecha{fecha de entrega}
%\grupo{Nro grupo}
\integrante{Agustina Ciraco}{630/06}{agusciraco@gmail.com}
\integrante{Alejandro Rebecchi}{15/10}{alejandrorebecchi@gmail.com}
\integrante{Maria Lara Gauder}{27/10}{marialaraa@gmail.com}
\integrante{Martin Heredia}{146/11}{martin.herediaf@gmail.com}

\include{templates}

\begin{document}
\pagestyle{myheadings}
\maketitle
%\markboth{Nombre materia}{Nombre TP}

\thispagestyle{empty}
\tableofcontents

%\setcounter{section}{-1}
\newpage

\section{Introducci\'on}

En este trabajo práctico se presentan Las arquitecturas del sistema correspondiente a la primera entrega (versión para escuela de villa urquiza) y la correspondiente al sistema pedido por el ministro, cuyo alcance es a nivel pais.
Se detallan Los atributos de calidad pedidos, junto con sus respectivos escenarios. Por otro lado se presenta una discusión sobre las diferencias entre las metodologías utilizadas en cada tp, y por último conclusiones.
\newpage
\section{Atributos de Calidad}
\subsection{Listado atributos de calidad pedidos}
\begin{itemize}
\item[Performance] Monitorear el estado de las campañas de manera agil. No se admiten demoras de ningún tipo .
\item[Disponibilidad] Fallas de comunicación durante transición %??
\item[Seguridad]Proteger datos y modificación (campañas y evaluación).
\item[Seguridad]Auditar
\item[Disponibilidad] Patagonia
\end{itemize}
\subsection{Escenarios de Atributos de calidad}
\begin{table}[H]
\centering
\begin{tabular}{ | p{5cm} | p{8cm} | p{1.5cm} | }
\hline
Clasificación & Caso de Uso & Horas H.\\ \hline \hline
\multirow{2}{5cm}{Usuarios} & Ingresando al sistema & 36 \\ \cline{2-3}
& Cargando usuario & 13 \\ \hline
%~ & Importando suscripciones& 13 \\ \hline
Eventos & Creando Evento & 7 \\ \hline
\multirow{2}{5cm}{Campañas} & Creando Campaña & 29 \\ \cline{2-3}
& Evaluando Campañas & 18 \\ \hline
\multirow{3}{5cm}{Administración de mensajes} & Enviando Mensajes & 32 \\ \cline{2-3}
& Recibiendo respuesta & 25 \\ \cline{2-3}
& Programando envio de mensajes & 16 \\ \hline
\multirow{4}{5cm}{Destinatarios} & Creando Destinatario & 14 \\  \cline{2-3}
& Modificando datos de Destinatario & 11 \\ \cline{2-3}
& Cancelando Suscripci\'on & 10 \\ \cline{2-3}
& Suscribiendo destinatario a una campaña & 10 \\ \hline
\multirow{3}{5cm}{Resultados} & Cargando Resultado de Campaña & 11 \\ \cline{2-3}
& Comparando Campañas & 14 \\ \cline{2-3}
& Eligiendo métrica & 5 \\ \hline
\multirow{4}{5cm}{Visualización} & Visualizando Campaña & 20 \\ \cline{2-3}
& Visualizando Evento & 20 \\ \cline{2-3}
& Visualizando Resultados & 20 \\ \cline{2-3}
& Visualizando Destinatario & 20 \\ \hline

\end{tabular}
\end{table}


\newpage

\section{Arquitecturas}
\subsection{Arquitectura TP1}
\subsection{Arquitectura TP2}
\subsection{Discusión de Arquitecturas}

\newpage
\section{Discusión Metodologías}
\subsection{Metodologías TP1}
\subsection{Metodologías TP2}
\subsection{Diferencias}

\newpage
\section{Conclusiones}

\end{document}