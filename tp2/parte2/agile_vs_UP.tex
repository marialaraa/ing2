\subsection{Agile vs UP}

El proceso de desarrollo de la aplicación se separó en dos etapas. En la primera, cuando se pensaba una aplicación de pequeña escala, se trabajó con metodologías ágiles, específicamente con SCRUM. Ésta propone realizar un proceso iterativo incremental separado en sprints cortos (generalmente menos de dos semanas), flexible, adaptable, manejando como unidad de trabajo user stories y sus tareas. Al inicio del proceso se cargan en el backlog todas las user stories asignandoles un esfuerzo y un valor para el cliente. Luego al inicial cada sprint se eligen entre las pendientes, las que tienen la mejor relación valor/esfuerzo. Esto tiene algunos problemas, ya que no permite ver con mucha claridad el estado actual del proyecto o proyecciones a futuro. Se centra principalmente en el sprint actual.

Por otro lado, en la segunda etapa, en la que se pensaba una aplicación de gran escala y se trabajo baja la modalidad de UP (proceso unificado), la cual propone un proceso también iterativo incremental, pero organizado en fases(inicio, elaboración, construccion y transicion), compuestas por iteraciones. Al iniciar, de forma similar a SCRUM, se definen todos los casos de uso, pero luego se asignan directamente a iteraciones, y éstas a su vez a sus etapas correspondientes. De esta forma se tiene una visualización mas completa de todo el proceso, y durante el desarrollo del mismo, se puede tener una noción mas clara de la etapa actual y la proyección a futuro. A su vez ésto limita un poco la flexibilidad del esquema, requiriendo modificaciones en el plan ante eventuales retrasos u otro tipo de inconvenientes.