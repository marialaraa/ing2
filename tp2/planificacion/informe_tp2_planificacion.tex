\documentclass[a4paper, 11pt]{article}
\usepackage{amsmath}
\usepackage{amsfonts}
\usepackage{amssymb}
\usepackage{caratula}
\usepackage[spanish, activeacute]{babel}
\usepackage[usenames,dvipsnames]{color}
\usepackage[width=15.5cm, left=3cm, top=2.5cm, height= 24.5cm]{geometry}
\usepackage{graphicx}
\usepackage[utf8]{inputenc}
\usepackage{listings}
\usepackage[all]{xy}
\usepackage{multicol}
\usepackage{subfig}

\usepackage{cancel}
\usepackage{float}
\usepackage{xcolor}
\usepackage{color,hyperref}

\usepackage{multirow} % para las tablas


%%%%%%%%%%%%%% ALGUNAS MACROS %%%%%%%%%%%%%%
% For \url{SOME_URL}, links SOME_URL to the url SOME_URL
\providecommand*\url[1]{\href{#1}{#1}}

% Same as above, but pretty-prints SOME_URL in teletype fixed-width font
\renewcommand*\url[1]{\href{#1}{\texttt{#1}}}

% Comando para poner el simbolo de Reales
\newcommand{\real}{\hbox{\bf R}}

\providecommand*\code[1]{\texttt{#1}}

%uso: \ponerGrafico{file}{caption}{scale}{label}
\newcommand{\ponerGrafico}[4]
{\begin{figure}[H]
	\centering
	\subfloat{\includegraphics[scale=#3]{#1}}
	\caption{#2} \label{fig:#4}
\end{figure}
}

%\renewcommand{\algorithmiccomment}[1]{\hfill #1}

%%%%%%%%%%%%%%%%%%%%%%%%%%%%%%%%%%%%%%%%%%%%

\materia{Ingeniería de Software II}

\titulo{Big Tiza}
%\fecha{fecha de entrega}
%\grupo{Nro grupo}
\integrante{Agustina Ciraco}{630/06}{agusciraco@gmail.com}
\integrante{Alejandro Rebecchi}{15/10}{alejandrorebecchi@gmail.com}
\integrante{Maria Lara Gauder}{27/10}{marialaraa@gmail.com}
\integrante{Martin Heredia}{146/11}{martin.herediaf@gmail.com}

\include{templates}

\begin{document}
\pagestyle{myheadings}
\maketitle
%\markboth{Nombre materia}{Nombre TP}

\thispagestyle{empty}
\tableofcontents

%\setcounter{section}{-1}
\newpage

\section{Introducción}
Debido al éxito del sistema AULA AL 2020, desarrollado para una escuela en Urquiza, se nos pidió ampliar el mismo para tener alcance nacional. Eso se debe a que notaron la eficiencia del sistema de avisos para generar mejoras el rendimiento de las campañas de comunicación y educación. Se deja de tener a los docentes y personales de la secretaría o dirección, para tener responsables de comunicación a nivel nacional, médicos sanitaristas, diseñadores de políticas culturales, empleados munincipales, entre otros. Lo mismo ocurre con los alumnos y padres, ya que ahora se podrá aplicar el sistema de envío a todas las personas que se deseen. Además, el SMS no será más el único canal de comunicación existente, sino que también se incorporarán nuevos. Es importante, también, mejorar el sistema de comparación de campañas y generación de resultados. 

A partir de la consigna entragada por la cátedra y de una reunión con el asistente del Ministro de Ciencias y Tecnología, el responsable técnico de ArSat, el representante de Unión Industrial Argnentina, el representante de Defensa al Consumidor y el dueño del sistema de abono de datos, se definieron los lineamientos a seguir. Se establecen los siguientes atributos en orden de prioridad descendente: 
\begin{itemize}
\item Disponibilidad
\item Performance
\item Seguridad
\item Certeza de Datos
\item Modificabilidad
\item Flexibilidad
\item Usabilidad
\end{itemize}

A continuación se presentará la parte de planificación del proyecto, que consiste en el plan, la lista de los casos de uso y el análisis de los riesgos.  


\newpage
\section{Enumeraci\'on casos de uso}


\begin{table}[H]
\centering
\begin{tabular}{ | p{5cm} | p{8cm} | p{1.5cm} | }
\hline
Clasificación & Caso de Uso & Horas H.\\ \hline \hline
Usuarios & Ingresando al sistema & 13 \\ \hline
%~ & Importando suscripciones& 13 \\ \hline
Eventos & Creando Evento & 13 \\ \hline
\multirow{2}{5cm}{Campañas} & Creando Campaña & 13 \\ \cline{2-3} 
& Evaluando Campañas & 13 \\ \hline
\multirow{3}{5cm}{Administración de mensajes} & Enviando Mensajes de campaña & 13 \\ \cline{2-3} 
& Recibiendo respuesta de campaña & 13 \\ \cline{2-3} 
& Programando envio de mensajes de campaña & 13 \\ \hline
\multirow{4}{5cm}{Destinatarios} & Creando Destinatario & 13 \\  \cline{2-3} 
& Modificando datos de suscripci\'on (Persona) & 13 \\ \cline{2-3} 
& Cancelando Suscripci\'on & 13 \\ \cline{2-3} 
& Suscribiendo a una campaña (Persona) & 13 \\ \hline 
\multirow{3}{5cm}{Resultados} & Cargando Resultado de Campaña & 13 \\ \cline{2-3} 
& Comparando Campañas & 13 \\ \cline{2-3} 
& Eligiendo métrica & 13 \\ \hline
\multirow{4}{5cm}{Visualización} & Visualizando Campaña & 13 \\ \cline{2-3} 
& Visualizando Evento & 13 \\ \cline{2-3} 
& Visualizando Resultados & 13 \\ \cline{2-3} 
& Visualizando Destinatario & 13 \\ \hline

\end{tabular}
\end{table}

Como se puede observar en la tabla, se presentan los casos de usos definidos para este sistema. A su vez para cada uno se define la cantidad de horas que llevaría completar cada uno de ellos, es decir, cumplir con todas las tareas que impliquen el funcionamiento del caso de uso. Además, se agrupan los casos de usos en 7 categorías diferentes. Para definir las mismas nos basamos en el diseño orientado a objetos realizado previamente. Por ejemplo, la categoria Usuario refiere a aquellos que deberán acceder al sistema para crear, modificar o eliminar instancias de ciertos objetos. Los Eventos y Campañas se relacionan con los que implican acciones sobre ese objeto. Además, la Administración de mensajes comprende los casos en los que se contempla del envío o recepción de mensajes. 
En el caso de Destinatario, se asignan todos los casos de usos relacionados con el objeto destinatario, es decir, la persona a quien se le enviará el aviso. Los Resultados son requeridos para poder ver la eficiencia de las campañas implementadas, por lo que entonces deberán implementarse diversos casos de usos relacionados. 
Por último, en Visualización hacemos referencia a aquellos casos de usos relacionados con la interfaz del sistema, sin importar a que objeto están representando. 

%Los casos se pueden agrupar....
%COMENTAR PORQUE LA AGRUPACION
%\begin{itemize}
%\item Casos de uso Visualizaci\'on.
%\item Casos de uso Administraci\'on de mensajes ()
%\item Casos de uso Campa\~nas.
%\item Casos de uso Resultados.
%\item Casos de uso Destinatarios.
%$\end{itemize}

\newpage

\section{Cronograma}
%JUSTIFICAR
\textbf{Qué factores tuvieron en cuenta para decidir la cantidad de iteraciones?\\
Cuánto dura cada iteración? \\
Cómo decidieron qué va en cada iteración?\\
Cómo influyeron los riesgos?}

Cada iteración presenta una duración de tres semanas. Esto decición fue tomada a partir de las fechas de comienzo (1/6) y primera entrega del proyecto (22/6) donde se deberá presentar la primera iteración del plan finalizada cumpliendo ciertos requisitos pedidos por los stakeholders. A su vez, se contemplaron lsa horas que pueden requerirse para el cumplimiento de cada caso de uso, definiendose así que con tres semanas se podrán cumplir todas las tareas definidas para cada iteración.

Para decidir que casos de usos se definirían en cada iteración, en primer lugar se armó una lista definiendo a los mismos por prioridad. En primer lugar se contempla el de ''Enviando Mensajes de Campaña'', ya que representa la base del sistema a implementar y a su vez, contiene la mayor cantidad de escenarios de riesgos con impacto mayor en el correcto funcionamiento en la implementeación. También, se pide como primer atributo de calidad el de Disponibilidad, es decir, que ante una falla del sistema, se deberá solucionar rápidamente el mismo para que el suscripto pueda recibir igualmente el aviso sin demoras. 
Por otro lado, se pide para el mes de julio poder demostrar el nivel de seguridad del sistema con respecto a los distintos objetos. Por lo que se decide desarrollar el caso de uso ''Creando Campaña'', ya que el mismo involucra varios requerimientos de seguridad. Lo mismo ocurre con el caso de uso ''Creando Destinatario''.
Luego, se continuó definiendo la lista para las demás iteraciones, basándonos fuertemente en las prioridades definidas en lo atributos de calidad. Finalmente, se dividieron en iteraciones de acuerdo al tiempo que puede tomar cada implementación, respetando las tres semanas por cada una. 



\subsection{Iteraciones}
\subsubsection{Primera Iteraci\'on:}
\begin{itemize}
\item Enviando Mensajes de Campaña
\item Creando Campaña
\item Creando Destinatario
\end{itemize}

\subsubsection{Segunda Iteraci\'on:}
\begin{itemize}
\item Cargando Resultado de Campaña
\item Programando envío de Mensajes de Campaña
\item Suscribiendo a una Campaña
\end{itemize}


\subsubsection{Tercera Iteraci\'on:}
\begin{itemize}
\item Visualizando Campaña
\item Visualizando Resultados
\item Visualizando Destinatario
\end{itemize}


\subsubsection{Cuarta Iteraci\'on:}
\begin{itemize}
\item Creando Evento
\item Visualizando Evento
\item Recibiendo respuesta de Campaña
\end{itemize}


\subsubsection{Quinta Iteraci\'on:}
\begin{itemize}
\item Eligiendo métrica
\item Comparando Campañas
\item Evaluando Campañas
\end{itemize}


\subsubsection{Sexta Iteraci\'on:}
\begin{itemize}
\item Cancelando Suscripción
\item Modificando datos de suscripción
\item Ingresando al sistema
\end{itemize}

\subsection{Detalle Primera Iteración}
A continuación se realiza una descripción a alto nivel de los casos de uso a desarrollar durante la primera iteración del proyecto. 
\begin{itemize}
\item \textbf{Enviando mensajes de campaña}: Se refiere a la acción de enviar los mensajes que son definidos en la campaña. Los mismos podrán ser enviados como SMS o utilizando alguna de las redes sociales (Facebook, Twitter, etc.). Se deberá garantizar que los mismos llegarán al destinatario de forma correcta, aplicándose otra tecnología en caso de existir una falla en el sistema. En este caso de uso se contempla únicamente el envío de mesajes, pero al comunicación es bilateral. 
\item \textbf{Creando campaña}: Se refiere al proceso de comienzo de una campaña, donde un usuario le define el título, las fechas, los suscriptores y los mensajes que se enviarán. Luego, se podrá modificar la misma, pero teniendo en cuenta quien tiene permisos y quien no. La seguridad del sistema es fundamental en este caso de uso. Los datos deben ser almacenados de forma correcta y contemplando posibles modificaciones. 
\item \textbf{Creando destinatario}: Se refiere al almacenamiento en el sistema de todas las peronas que sean cargadas. Se debe tener en cuenta que se estarán manipulando grandes volúmenes de datos. A su vez, es importante que a los mismos solo puedan acceder aquellos usuarios habilitados previamente. Esta información será utilizado luego en la creación de las campañas, ya que pasarán a ser suscriptores. Los destinatarios deberán estar compuestos por un Nombre, Apellido y Celular. A su vez, se deberá permitir agregar otros campos diferentes para cada destinatario. 
\end{itemize}

\newpage
\section{Riesgos}

Los casos de uso que presentan escenarios de riesgo son:
\begin{itemize}
\item Modificando Datos teléfono
\item Enviando Mensajes de campaña
\item Recibiendo respuesta de campaña
\item Programando envio de mensajes de campaña
\item Creando Campaña
\item Creando Evento
\item Cargando Resultado de Campaña
\end{itemize}
Los mismos se pueden dividir en dos tipos de riesgo principales. El de índole de problemas de conectividad (mensajes) y los de la índole persistencia (bases de datos).

\subsection{Riesgo Sistema de envio de mensajes SMS}
\begin{itemize}
\item \textbf{Descripci\'on:} Dado que la Red telefónica (para SMS) es propensa a fallas en el envio de mensajes, es posible enviar mesajes por otro medio (facebook twitter)
\item \textbf{Probabilidad:} Alta (Patagonia)
\item \textbf{Impacto:} Alto
\item \textbf{Exposici\'on:} Alta
\item \textbf{Mitigaci\'on:} Generar pruebas de cortes temporario para disparar el reenvio.
\item \textbf{Plan de Contingencia:} Enviar drones para generar conectividad. Reenvio de mensajes
\end{itemize}


\subsection{Riesgo Sistema de envío mensajes Web}
\begin{itemize}
\item \textbf{Descripci\'on:} Dado que la Red de Internet es propensa a fallas es posible enviar mesajes por otro medio (ej: Correo Argentino) 
\item \textbf{Probabilidad:} Media 
\item \textbf{Impacto:} Alto
\item \textbf{Exposici\'on:} Alta
\item \textbf{Mitigaci\'on:} Hacer pruebas de env\'io por correo para estimar tiempos de respuesta. 
\item \textbf{Plan de Contingencia:} Contactar a Correo Argentino para el envío de los mensajes de la campa\~na.
\end{itemize}

\subsection{Riesgo Sistema de envío de mensajes}
\begin{itemize}
\item \textbf{Descripci\'on:} Dado que el scheduler de la programaci\'on del env\'io de mensajes es propenso a fallas (por ejemplo: reloj, bug, etc.) es posible enviar los mensajes de forma manual.
\item \textbf{Probabilidad:} Baja
\item \textbf{Impacto:} Alto
\item \textbf{Exposici\'on:} Media
\item \textbf{Mitigaci\'on:} Hacer pruebas sostenidas durante distintos per\'iodos de tiempo (por ejemplo: una hora, un d\'ia, una semana).
\item \textbf{Plan de Contingencia:} Ejecutar el env\'io de mensajes de forma manual.
\end{itemize}

\subsection{Riesgo Servidor con Base de Datos centralizada}
\begin{itemize}

\item \textbf{Descripci\'on:} Dado que la conección a la base puede fallar generando que las transacciones aborten, por lo que es propenso a fallas.
\item \textbf{Probabilidad:} Media
\item \textbf{Impacto:} Medio
\item \textbf{Exposici\'on:} Medio
\item \textbf{Mitigaci\'on:} Hacer pruebas de concurrencia y stress.
\item \textbf{Plan de Contingencia:} Generar archivo local con transacciones a realizar y reintentar en un determinado tiempo.
\end{itemize}

%~ \newpage




\subsection{Detalle de la primera iteraci\'on}
\begin{itemize}
\item Identificador de iteraci\'on: E01
\item Tipo de iteraci\'on: Elaboracion
\item Tareas:
\begin{itemize}
\item{E01-T01} Diseño conceptual del sistema
\item{E01-T02} Realizción del WBS
\item{E01-T03} Análisis de Riesgos
\item{E01-T04} Reafinamiento de Objetivos y Requerimientos
\item{E01-T05} Reconocimiento de casos de uso
\item{E01-T06} Priorizaci\'on de casos de uso 
\item{E01-T07} Estimación de tiempos de casos de uso
\item{E01-T08} Análisis de atributos de calidad del sistema
\item{E01-T09} Diseño de arquitectura del sistema
\item{E01-T10} Investigacion de tecnolog\'ias
\item{E01-T11} Elección de tecnologías
\item{E01-T12} Realización de tareas de caso de uso CU01 - Enviando mensajes
\item{E01-T13} Realización de tareas de caso de uso CU02 - Creando Campa\~na
\item{E01-T14} Realización de tareas de caso de uso CU03 - Creando Destinatario
\item{E01-T15} Test de arquitectura del sistema a partir de un test de stress
\end{itemize}
\end{itemize}

\subsection{Lista de tareas relacionadas con casos de uso de primera iteraci\'on}
\subsubsection{Enviando mensajes}
\begin{itemize}
\item{CU\#01-T01} Configurar ambiente desarrollo
\item{CU\#01-T02} Investigar API facebook, twitter, sms
\item{CU\#01-T03} Investigar Seguridad
\item{CU\#01-T04} Implementar Mock m\'odulo de recepci\'on de mensajes
\item{CU\#01-T05} Dise\~no protocolo comunicaci\'on API 
\item{CU\#01-T06} Dise\~no protocolo comunicaci\'on Sistema  (como lee los mensajes con destinatarios a enviar)
\item{CU\#01-T07} Implementar protocolo comunicaci\'on API
\item{CU\#01-T08} Implementar protocolo comunicaci\'on sistema
\item{CU\#01-T09} Implementar caso de uso
\end{itemize}

\subsubsection{Creando Campa\~na}
\begin{itemize}
\item{CU\#01-T01} Configurar ambiente desarrollo
\item{CU\#01-T02} Modelo Entidad Relaci\'on
\item{CU\#01-T03} Diagrama Entidad Relaci\'on
\item{CU\#01-T04} Configurar la base de datos (para campa\~na)
\item{CU\#01-T05} Dise\~no campa\~na
\begin{itemize}
\item{CU\#01-T05 - st01} Dise\~no Mensaje
\item{CU\#01-T05 - st02} Dise\~no suscripci\'on
\item{CU\#01-T05 - st03} Dise\~no capa seguridad
\end{itemize}
\item{CU\#01-T06} Implementaci\'on Caso de uso
\begin{itemize}
\item{CU\#01-T06 - st01} Dise\~no Mensaje
\item{CU\#01-T06 - st02} Dise\~no suscripci\'on
\item{CU\#01-T06 - st03} Dise\~no capa seguridad
\end{itemize}
\item{CU\#01-T07} testing
\end{itemize}

\subsubsection{Creando Destinatario}
\begin{itemize}
\item{CU\#01-T01} Investigar Formas de almacenamiento (Big Data)
\item{CU\#01-T02} Investigar formas de seguridad de acceso a los datos
\item{CU\#01-T03} Dise\~no caso de uso
\item{CU\#01-T04} Implementaci\'on caso de uso
\begin{itemize}
\item{CU\#01-T04 - st01} Integraci\'n con almacenamiento
\item{CU\#01-T04 - st02} Implementaci\'on comunicaci\'on con almacenamiento
\end{itemize}
\item{CU\#01-T05} Testing
\end{itemize}
\subsection{Estimaci\'on de tiempo en horas hombre}

%grafo dependencia de tareas.
%Gant
\end{document}
