\subsection{Agile vs UP}

El proceso de desarrollo de la aplicaci\'on se separ\'o en dos etapas. En la primera, cuando se pensaba una aplicaci\'on de peque\~na escala, se trabaj\'o con metodolog\'ias \'agiles, espec\'ificamente con SCRUM. Ésta propone realizar un proceso iterativo incremental separado en sprints cortos (generalmente menos de dos semanas), flexible, adaptable, manejando como unidad de trabajo user stories y sus tareas. Al inicio del proceso se cargan en el backlog todas las user stories asignandoles un esfuerzo y un valor para el cliente. Luego al iniciar cada sprint se eligen entre las pendientes, las que tienen la mejor relaci\'on valor/esfuerzo. Esto tiene algunos problemas, ya que no permite ver con mucha claridad el estado actual del proyecto o proyecciones a futuro. Se centra principalmente en el sprint actual.

Por otro lado, en la segunda etapa, en la que se pensaba una aplicaci\'on de gran escala y se trabajo baja la modalidad de UP (proceso unificado), la cual propone un proceso tambi\'en iterativo incremental, pero organizado en fases (inicio, elaboraci\'on, construccion y transicion), compuestas por iteraciones. Al iniciar, de forma similar a SCRUM, se definen todos los casos de uso, pero luego se asignan directamente a iteraciones, y estas a su vez a sus etapas correspondientes. De esta forma se tiene una visualizaci\'on mas completa de todo el proceso, y durante el desarrollo del mismo, se puede tener una noci\'on mas clara de la etapa actual y la proyecci\'on a futuro. A su vez \'esto limita un poco la flexibilidad del esquema, requiriendo modificaciones en el plan ante eventuales retrasos u otro tipo de inconvenientes.