\documentclass[a4paper, 11pt]{article}
\usepackage{amsmath}
\usepackage{amsfonts}
\usepackage{amssymb}
\usepackage{caratula}
\usepackage[spanish, activeacute]{babel}
\usepackage[usenames,dvipsnames]{color}
\usepackage[width=15.5cm, left=3cm, top=2.5cm, height= 24.5cm]{geometry}
\usepackage{graphicx}
\usepackage[utf8]{inputenc}
\usepackage{listings}
\usepackage[all]{xy}
\usepackage{multicol}
\usepackage{subfig}

\usepackage{cancel}
\usepackage{float}
\usepackage{xcolor}
\usepackage{color,hyperref}

\usepackage{multirow} % para las tablas


%%%%%%%%%%%%%% ALGUNAS MACROS %%%%%%%%%%%%%%
% For \url{SOME_URL}, links SOME_URL to the url SOME_URL
\providecommand*\url[1]{\href{#1}{#1}}

% Same as above, but pretty-prints SOME_URL in teletype fixed-width font
\renewcommand*\url[1]{\href{#1}{\texttt{#1}}}

% Comando para poner el simbolo de Reales
\newcommand{\real}{\hbox{\bf R}}

\providecommand*\code[1]{\texttt{#1}}

%uso: \ponerGrafico{file}{caption}{scale}{label}
\newcommand{\ponerGrafico}[4]
{\begin{figure}[H]
	\centering
	\subfloat{\includegraphics[scale=#3]{#1}}
	\caption{#2} \label{fig:#4}
\end{figure}
}

%\renewcommand{\algorithmiccomment}[1]{\hfill #1}

%%%%%%%%%%%%%%%%%%%%%%%%%%%%%%%%%%%%%%%%%%%%

\materia{Ingeniería de Software II}

\titulo{Big Tiza}
%\fecha{fecha de entrega}
%\grupo{Nro grupo}
\integrante{Agustina Ciraco}{630/06}{agusciraco@gmail.com}
\integrante{Alejandro Rebecchi}{15/10}{alejandrorebecchi@gmail.com}
\integrante{Maria Lara Gauder}{27/10}{marialaraa@gmail.com}
\integrante{Martin Heredia}{146/11}{martin.herediaf@gmail.com}

\include{templates}

\begin{document}
\pagestyle{myheadings}
\maketitle
%\markboth{Nombre materia}{Nombre TP}

\thispagestyle{empty}
\tableofcontents

%\setcounter{section}{-1}
\newpage

\section{Introducción}


\newpage
\section{Enumeraci\'on casos de uso}


\begin{table}[H]
\centering
\begin{tabular}{ | p{10cm} | p{1cm} |}
\hline
Caso de Uso & Horas H.\\ \hline \hline
Creando Destinatario \\ \hline \hline
Cancelando Suscripci\'on& 13 \\ \hline
Suscribiendo a una campaña& 13 \\ \hline
%~ Importando suscripciones& 13 \\ \hline
Modificando datos de suscripci\'on (Persona)& 13 \\ \hline
Enviando Mensajes de campaña& 13 \\ \hline
Recibiendo respuesta de campaña& 13 \\ \hline
Programando envio de mensajes de campaña& 13 \\ \hline
Creando Campaña& 13 \\ \hline
Creando Evento& 13 \\ \hline
Cargando Resultado de Campaña& 13 \\ \hline
Eligiendo métrica& 13 \\ \hline
Comparando Campañas& 13 \\ \hline
Evaluando Campañas& 13 \\ \hline
Visualizando Campaña& 13 \\ \hline
Visualizando Evento& 13 \\ \hline
Visualizando Resultados & 13 \\ \hline
Visualizando Datos (Persona) & 13 \\ \hline
Ingresando al sistema & 13 \\ \hline

\end{tabular}
\end{table}
Los casos se pueden agrupar....
\begin{itemize}
\item Casos de uso Visualizaci\'on.
\item Casos de uso Administraci\'on de mensajes ()
\end{itemize}
Qué factores tuvieron en cuenta para decidir la cantidad de iteraciones?
Cuánto dura cada iteración?
Cómo decidieron qué va en cada iteración?
Cómo influyeron los riesgos?

\section{Riesgos}

Los casos de uso con riesgo son:
\begin{itemize}
\item Modificando Datos teléfono
\item Enviando Mensajes de campaña
\item Recibiendo respuesta de campaña
\item Programando envio de mensajes de campaña
\item Creando Campaña
\item Creando Evento
\item Cargando Resultado de Campaña
\end{itemize}
Los mismos se pueden dividir en dos tipos de riesgo principales. El de índole de problemas de conectividad (mensajes) y los de la índole persistencia (Bases de datos)

\subsection{Riesgo 1}
\begin{itemize}

\item \textbf{Descripci\'on:} Dado que la Red telefónica (para SMS) es propensa a fallas en el envio de mensajes, es posible enviar mesajes por otro medio (facebook twitter)
\item \textbf{Probabilidad:} Alta (Patagonia)
\item \textbf{Impacto:} Alto
\item \textbf{Exposici\'on:} Alta
\item \textbf{Mitigaci\'on:} Generar pruebas de cortes temporario para disparar el reenvio.
\item \textbf{Plan de Contingencia:} Enviar drones para generar conectividad. Reenvio de mensajes
\end{itemize}


\subsection{Riesgo 1}
\begin{itemize}

\item \textbf{Descripci\'on:} Dado que la Red telefónica (para SMS) es propensa a fallas es posible enviar mesajes por otro medio (facebook twitter)
\item \textbf{Probabilidad:}
\item \textbf{Impacto:}
\item \textbf{Exposici\'on:}
\item \textbf{Mitigaci\'on:}
\item \textbf{Plan de Contingencia:}
\end{itemize}

\subsection{Riesgo 1}
\begin{itemize}

\item \textbf{Descripci\'on:} Dado que la Red telefónica (para SMS) es propensa a fallas es posible enviar mesajes por otro medio (facebook twitter)
\item \textbf{Probabilidad:}
\item \textbf{Impacto:}
\item \textbf{Exposici\'on:}
\item \textbf{Mitigaci\'on:}
\item \textbf{Plan de Contingencia:}
\end{itemize}

\subsection{Riesgo 1}
\begin{itemize}

\item \textbf{Descripci\'on:} Dado que la conección a la base puede fallar generando que las transacciones aborten, por lo que es propenso a fallas.
\item \textbf{Probabilidad:} Media
\item \textbf{Impacto:} Medio
\item \textbf{Exposici\'on:} Medio
\item \textbf{Mitigaci\'on:} Generar archivo local con transacciones a realizar.
\item \textbf{Plan de Contingencia:} 
\end{itemize}

\newpage

\section{Cronograma}


\subsection{Detalle de la primera iteraci\'on}

\subsection{Lista de tareas relacionadas con casos de uso de primera iteraci\'on}

\subsection{Estimaci\'on de tiempo en horas hombre}

%grafo dependencia de tareas.
%Gant
\end{document}
