\subsection{Programming in the Large vs Programming in the Small}
En cuanto al desarrollo de la implementación del proyecto, dada la diferencia de magnitudes, para las distintas etapas utilizamos dos tipos de técnicas bastante diferentes. Para el primer caso, el de menor escala, utilizamos la técnica de programming in the small, la cual se centra en los detalles finos del diseño del modelo de datos del negocio. Se buscaespecìficamente que el código modele de forma semánticamente correcta las características de los objetos de interés del mundo real desde la perspectiva del negocio de interés.

Por su parte, para el segundo caso, el de mayor escala, se utilizó la técnica de programming in the large, la cual se centra en la arquitectura general del proyecto, sin hacer tanto incapié en detalles del código, sino que se manejan directamente modulos de código, componentes y conectores, e incluso componentes de hardware. Ésto nos brinda una visión más general del proyecto, mostrando su interacción con componentes externos y a su vez dando una visiòn más organizativa en general.