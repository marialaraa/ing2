\section{Conclusiones}

En este trabajo pr\'actico se compararon tanto arquitecturas como metodolog\'ias utilizadas para el primer trabajo pr\'actico de la materia con respecto a lo pedido en este trabajo.
En principio una de las grandes diferencias entre el trabajo pr\'actico 1 (de ahora en m\'as se refiere a este como TP1) con respecto a este, es el volumen de datos, usuarios y servicios de mensajer\'ia. Esto lleva a modelar de forma diferente cada sistema, ya que por ej\'emplo, en este trabajo, se tuvieron en cuenta a la hora de planificar y manejar tiempos, los riesgos, ya que es un proyecto a realizar en un plazo estimado de un a\~no. Por ello para evitar atrasos no contemplados, se trabaja en identificar riesgos y dependencias de tareas que puedan llevar a una demora indeseada. En cambio, en el TP1 al ser un proyecto de tan solo meses, se puede estimar con solo el dise\~no, aunque tambi\'en es una buena pr\'actica la identificaci\'on de dependencias de tareas.
Otra diferencia importante entre el TP1 y \'este es el manejo de Big Data. Esto se vio reflejado en la evaluaci\'on de resultados de campa\~nas, ya que en el TP1, dicha evaluaci\'on se realizaba en su totalidad y en poco tiempo. Sin embargo en el presente trabajo se trabajan con resultados parciales ya que el volumen de datos que se espera recibir es mucho mayor y eso implican demoras a la hora de evaluar la totalidad de datos. Como adem\'as se desea poder observar la campa\~na y tener lo m\'as actualizado posible el resultado, no se podr\'ia lograr esperando el resultado final.
En cuanto a las diferentes metodolog\'ias entre los trabajos mencionados, se not\'o que las metodolog\'ias aplicadas en el TP1 son m\'as din\'amicas ya que se manejan tiempos m\'as cortos, mientas que se notaron procesos m\'as largos en este trabajo, por lo comentado anteriormente en cuanto a, por ej\'emplo, planificaci\'on. Sin embargo, una similitud que se not\'o, aunque no se lleg\'o a aplicar, es que una vez definidas la planificaci\'on, arquitectura y se puede comenzar con el desarrollo del sistema de este trabajo pr\'actico, se pueden combinar metodolog\'ias de trabajo vistas en el TP1, como Scrum, y el manejo de Sprints.