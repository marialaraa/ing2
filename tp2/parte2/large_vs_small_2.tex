\subsection{Programming in the Large vs Programming in the Small}
En cuanto al desarrollo de la implementaci\'on del proyecto, dada la diferencia de magnitudes, para las distintas etapas utilizamos dos tipos de t\'ecnicas bastante diferentes. 

Para el primer caso, el de menor escala, utilizamos la t\'ecnica de programming in the small, la cual se centra en los detalles finos del dise\~no del modelo de datos del negocio. Se busca específicamente que el c\'odigo modele de forma sem\'anticamente correcta las caracter\'isticas de los objetos de inter\'es del mundo real desde la perspectiva del negocio de inter\'es.

Por su parte, para el segundo caso, el de mayor escala, se utiliz\'o la t\'ecnica de programming in the large, la cual se centra en la arquitectura general del proyecto, sin hacer tanto incapi\'e en detalles del c\'odigo, sino que se manejan directamente modulos de c\'odigo, componentes y conectores, e incluso componentes de hardware. Ésto nos brinda una visi\'on m\'as general del proyecto, mostrando su interacci\'on con componentes externos y a su vez dando una visi\'on m\'as organizativa en general.