\section{Conclusiones}

En este trabajo práctico se compararon tanto las arquitecturas como las metodologías utilizadas para el primer trabajo práctico de la materia con respecto a lo pedido en este.
En principio una de las grandes diferencias entre el trabajo práctico 1 (de ahora en más se refiere a este como TP1) con respecto al actual, es el importante aumento del volumen de datos, usuarios y incorporación de servicios de mensajería. 
Esto lleva a modelar de forma diferente cada sistema. Por ejemplo, en el proyecto de Big Tiza, se tuvieron en cuenta a la hora de planificar y manejar tiempos, los riesgos. Esto se debe a que es un proyecto a realizar en un plazo estimado de un año. Para evitar atrasos no contemplados, se trabaja inicialmente en identificar riesgos y dependencias de tareas que puedan llevar a una demora indeseada. En cambio, en el TP1 al ser un proyecto de tan solo meses, se puede estimar con sólo el diseño, la identificación de dependencias de tareas. \\
Otra diferencia importante entre el proyecto de una escuela y el actual, es el manejo de Big Data. Esto se vio reflejado en la evaluación de resultados de campañas, ya que en el TP1 dicha evaluación se realizaba en su totalidad y en poco tiempo al finalizar la campaña. Sin embargo, en el presente trabajo se analizan resultados parciales ya que el volumen de datos que se espera recibir es mucho mayor. Esto último implica demoras a la hora de evaluar la totalidad de los datos. Por el requerimiento de poder visualizar las estadísticas de las campañas en tiempo real, se debió incorporar un componente que procese resultados constantemente y genere aproximaciones de las mismas. \\
En cuanto a las diferentes metodologías entre los trabajos mencionados, se notó que las aplicadas en el TP1 son más dinámicas ya que se manejan tiempos más cortos, mientas que se notaron procesos más largos en este trabajo. Sin embargo, al trabajar a distintas escalas se podría combinar ambas técnicas en un mismo proyecto, por ejemplo utilizando la metodología SCRUM en un componente en particular. 

